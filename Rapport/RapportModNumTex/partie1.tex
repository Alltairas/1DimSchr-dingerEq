\section{Principe physique et méthodes de résolution}

\subsection{Equation de schrodinger}
L’équation de Schrödinger tire son nom du physicien Erwin Schrödinger qui a créé cette équation en 1925 pour laquelle il a obtenu le prix Nobel de physique en 1933. L’équation de Schrödinger est une équation d’onde qui tient compte à la fois de l’énergie non relativiste et la quantification de celle-ci. L’équation de Schrödinger est une équation fondamentale de la physique quantique, elle décrit l’évolution dans l’espace et le temps d’une particule massive non relativiste. Pour notre projet nous nous intéresserons à l’équation de Schrödinger stationnaire\cite{texier2015mecanique}, c’est-à-dire que l’équation sera indépendante du temps. Nous nous intéresserons qu’a une seule dimension l’équation dépendra donc que d’une seul direction ici x.

\begin{equation} \label{eq:schrodinger}
 \frac{-\hbar^2}{2m}\frac{d^2\psi(x)}{dx^2 }+V(x)\psi(x)=E\psi(x)
\end{equation}
\begin{center}
$ V(x) = \left\{
    \begin{array}{ll}
       f(x) & \mbox{si}\ 0<x<a \\
       0 & \mbox{sinon.}
    \end{array}
\right.$
\end{center}

\subsection{Unité système atomique}

Pour résoudre l’équation de Schrödinger nous allons nous placer dans le système d'unités atomiques \cite{frwiki:193318984}et non pas dans le système internationale l'équivalence entre les deux système est donné dans le tableau \ref{tab1}. Le système d’unités atomiques est mieux adapté car nous travaillons sur de la physique quantique. Les unités atomiques sont défini en posant la constantes de planck réduite $\hbar$, la masse de l’électron $m_{e}$  et la constante de l'électrostatique $\frac{e^2}{4\pi\epsilon_{0}}$égale à 1.


\begin{table}[!ht]
\centering
\begin{tabular}{|l|l|l|l|}
\hline Grandeur & Nom &  Valeur SI  \\
\hline  Masse & masse de l'électron & 9,109 382 6× $10^{-31}$ kg  \\
\hline Longueur & rayon de Bohr & 5,291 772 109 2× $10^{-11}$ m\\ 
\hline Énergie & hartree & 4,359 744 17 × $10^{-18} $J \\
\hline
\end{tabular}
\caption{Equivalence SI et unité atomique}
\label{tab1}
\end{table}


\subsection{Méthodes de résolution utilisées}
En dehors du potentiel l'équation \ref{eq:schrodinger} devient :

\begin{equation} \label{eq:schrodinger2}
\frac{d^2\psi(x)}{dx^2 }=\frac{-2mE}{\hbar^{2}}\psi(x)
\end{equation}

\begin{equation} \label{eq:schrodinger3}
\frac{d^2\psi(x)}{dx^2 }=-k^{2}\psi(x)
\end{equation}

\begin{center}
$k=\sqrt{\frac{2mE}{\hbar^{2}}}$
\end{center}

La solution générale de l'équation de Schrödinger est $\psi(x)=Ae^{ikx}+Be^{-ikx}$.

L'équation de Schrödinger admettra différentes solutions selon la région du potentiel dans laquelle elle se trouve. Pour $x<0$ l'équation sera de la forme de la solution générale, dans la région $II$ on cherchera la solution numériquement. Enfin pour $x>a$ la propagation dans le sens des x négatif est impossible donc la fonction d'onde sera de la forme  $\psi(x)=Ce^{ikx}$.
\newpage 
\begin{center}
$\psi_{I}(x)=Ae^{ikx}+Be^{-ikx}\\$
$\psi_{II}(x)=clacule \ numérique\\$
$\psi_{III}(x)=Ce^{ikx}\\$
\end{center}


Notre objectif est de trouver les coefficients de réflexion et de transmission, c'est à dire la probabilité pour qu'une particule passe à travers le potentiel :

\begin{equation} \label{eq:defcoeff}
R=|\frac{B}{A}|^{2}\qquad T=|\frac{C}{A}|^{2}\qquad R+T=1
\end{equation}
\subsubsection{Méthode 1}

Pour la méthode 1 on fixe la constante C=1. On impose également la continuité des dérivées de la fonction d'onde en 0 et en a.
\begin{center}
$\psi'_{I}(0)=\psi'_{II}(0)$\\
$\psi'_{II}(a)=\psi'_{III}(a)$
\end{center}
En reprenant l'équation de Schrödinger du début \ref{eq:schrodinger} et en utilisent la formule de la dérivée seconde discrète\,\footnote{$\frac{d^{2}f(x)}{dx^{2}}=\frac{f(x+h)+f(x-h)-2f(x)}{h^{2}}$}. On obtient une équation \ref{eq:numregion2} qui permet de calculer successivement toutes les valeurs de la fonction d'onde dans la région $II$ afin d'obtenir la valeur de $\psi(0)$ et  la valeur de $\psi(pas)$. En utilisant la formule de la dérivée discrète\,\footnote{$\frac{df(x)}{dx}=\frac{f(x+h)+f(x-h)}{h}$} et les valeurs de $\psi$ on trouve la dérivée de la fonction d'onde en 0 ($\psi'_{II}(0)$).

\begin{equation} \label{eq:numregion2}
\psi(x-pas)=-\psi(x+pas)+(\frac{2mpas^{2}}{\hbar^{2}}(V(x)-E)+2)\psi(x)
\end{equation}

Avec la continuité des dérivées on obtient le système d'équation suivant qui permet de déterminer les valeurs des constantes A et B, et ainsi de déterminer les deux coefficients recherchés.
\begin{center}
$ \left\{
    \begin{array}{ll}
       A+B= \psi_{II}(0) \\
       ikA+ikB=\psi'_{II}(0) 
    \end{array}
\right.$
\end{center}
\subsubsection{Méthode 2}
Pour la méthode 2 on fixe la constante A=1.Puis on choisie une valeur de B aléatoire ($B_{i}$) puis on utilise l'équation  \ref{eq:numregion2} pour calculer successivement les valeurs de la fonction d'onde dans la région $II$ afin d'obtenir la valeur de$\psi_{II}(a)$. On trouve alors une valeur de C, on utilise de nouveau l'équation de résolution numérique de la fonction d'onde pour trouver la valeur de $\psi_{II}(0)$. On trouve donc une nouvelle valeur de B ($B_{f}$), on cherche à se que la valeur de $B_{i}$soit identique à la valeur de $B_{f}$ pour cela on utilise un algorithme de minimisation dans notre cas l'algorithme de descente de gradient. La minimisation permet donc de déterminer les valeurs de B et de C, et ainsi les coefficients de transmission et de réflexion.

\subsubsection{Vérification fonctionnement}

Afin de vérifier que le calcule des coefficients de transmission et de réflexion sont correct on utilise la solution analytique connue \cite{antoine2022introduction}{}de la barrière de potentiel simple. 
\newpage 
$E<V_{0}$
\qquad
$\alpha=\sqrt{\frac{2m}{\hbar}(V_{0}-E}$
\begin{equation} \label{eq:analityquee<v}
T=\frac{1}{1+\frac{V_{0}^{2}sinh^{2}(\alpha a)}{4E(V_{0}-E)}}
\end{equation}

$E=V_{0}$
\begin{equation} \label{eq:analityquee=v}
T=\frac{1}{1+\frac{V_{0}a^{2}\hbar^{2}}{2}}
\end{equation}

$E>V_{0}$
\qquad
$k=\sqrt{\frac{2mV_{0}a^{2}}{\hbar}(1-\frac{E}{V_{0}}}$
\begin{equation} \label{eq:analityquee>v}
T=\frac{1}{1+\frac{V_{0}^{2}sin^{2}(k)}{4E(E-V_{0})}}
\end{equation}
\\

\begin{table}[!ht]
\centering
Paramètres utilisés $E=1.2$,\ $a=1$,\ $m=1$,\ $V_{0}=1$,\ $N=20000$\\
\begin{tabular}{|l|l|l|}
\hline   & Méthode 2 & Analytique  \\
\hline  Méthode 1 &   1.0917944672395e-05 & 2.70045e-05 \\
\hline Méthode 2 &   &1.60866e-05\\ 
\hline
\end{tabular}
\caption{Écarts entre les méthodes et solution analytique}
\label{tab2}
\end{table}