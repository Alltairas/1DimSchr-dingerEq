\section*{Introduction}
\addcontentsline{toc}{section}{Introduction}

\par Notre projet porte sur la résolution de l’équation  de Schrödinger stationnaire à une dimension, afin de déterminer les coefficients de réflexion et de transmission d’une particule à travers  un potentiel. Notre programme va résoudre l’équation de Schrödinger numériquement par deux méthodes. Pour vérifier la validité de la résolution numérique notre programme comparera le résultat à la valeur analytique connue pour une barrière de potentiel simple. Dans un premier temps nous reviendrons sur l’équation de Schrödinger, son principe et son utilisation, et nous présenterons les deux méthodes de résolution. Ensuite nous présenterons des résultats pour différents potentiels dont on ne connaît pas la solution analytique. Pour finir nous présenterons des résultats de la variation des différents paramètres, comme la masse du particule, la largeur du potentiel, pour une barrière de potentiel.

