\section*{Conclusion}
\addcontentsline{toc}{section}{Conclusion}

\par Pour conclure, nous avons modélisé plusieurs types de barrières de potentiels et réalisé le calcul numérique de la fonction d'onde et ainsi trouver les coefficients de réflexion et de transmission pour différentes énergies et masses. Par l'usage de l'interface graphique GNUplot, nous avons réalisé les courbes de Re($\Psi$(x)) et observé la capture de particule avec le phénomène de résonance. Il nous à été possible de montrer que à certaine énergie il est possible d'obtenir la transparence du potentiel pour avoir un coefficient de transmission égale à 1. La visualisation de l'effet tunnel permet aussi de comprendre le fonctionnement du microscope à effet tunnel.
Enfin pour ce qui pourrait êtres intéressant pour d'autre développement du programme, la résolution de l'équation de Schrödinger à plusieurs dimensions.
